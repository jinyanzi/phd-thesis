\section{Introduction}
\label{sec:scene-intro}

For a traffic camera mounted at a static location, objects in the recorded videos move along the road geometry with a regular pattern.
Those movement patterns could be useful for video analysis.
For example, objects that do not follow the usual movement pattern are likely to be anomalous; therefore, they are worth special attention.
Besides, statistics on different movements provide analysis on a finer granularity. 
For example, \gls{idot} keeps track of the vehicle counts in different moving directions, for traffic cameras all across Illinois.

It is usually intuitive for a human to identify the movement patterns after watching such a video for a while.
However, human's interpretation lacks a mathematical representation and may vary among different people.
Also, people are prone to making mistakes when dealing with complex videos; their performance may tend to decrease with longer working hours.

As a result, it is essential to automate this process for fast and large scale processing. 
Some researchers are working on clustering existing trajectories and obtain a semantic representation of the scene \cite{tung2011goal,xu2015unsupervised}.
However, such a method is usually sensitive to noises. 
More importantly, it is often hard to get clean trajectories via object tracking algorithms.
An additional step of data cleaning is necessary to make the clustering method work.
On the other hand, compared with object-level trajectories across multiple frames, pixel movement on individual frames is a more robust input for the scene understanding problem.
It excludes the interference of object interaction and forms a lower-level summary of the scene.
In this chapter, we apply a non-parametric clustering method \cite{wang2009unsupervised} to learn the vehicles' movement pattern in an unsupervised manner.
The resulting clusters have a mathematical representation in the form of a set of distributions; therefore, they are easier to interpret in the subsequent processing.