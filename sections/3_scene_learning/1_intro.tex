\section{Introduction}
\label{sec:scene-intro}

%  P1 background & motivation. “As traffic cameras become more prevalent…”
%  P2 related work “Earlier efforts in this direction have…”
%  P3 our work “We present … ” followed by “Our primary contributions are as follows: “
%   - bulleted list
%  P4 eval: “Our evaluation on datasets/benchmarks X, Y, Z, demonstrates…”
%  P5 outline: “The remainder of this paper is structured as follows…"
%%%%%%% Vision application requires human input, not automatic enough. %%%%%%% 
Transportation videos are nowadays captured by millions of cameras installed at local and national highways and streets\footnote{The Illinois Department of Transportation has a huge database of 24-hour traffic videos all across Illinois.}.
The analysis of such videos are critical for traffic volume monitoring, peak hour and congestion patterns discovery, tracing cars of criminals and stolen cars, highway toll management, among many others.
For decades people have been trying to extract information from transportation surveillance videos.
For example, one can hire a large number of human workers to watch and mark up any subregions, frames, or clips, that contain information of interest.
This practice is labor-intensive and not scalable,
as the workers need to pause from time to time and carefully analyze each individual object frame-wise.
Even with high detection precision, recall can be low due to the bottleneck inexpensive human labor and processing bandwidth.

%One can map the image to the real-world coordinates either by optical geometry or hand-annotated metrics. 
%With camera specifications or hand-annotated measurements, real-world coordinates, such as vehicle height and width \cite{cheng2011intelligent} or traffic statistics \cite{corral2017slot}, can be restored.
%While most methods directly work on the image coordinates, with some predefined semantic representation, either as input or constraints, surveillance tasks such as vehicle tracking and counting \cite{lessard2016countingapp} could be completed.
% list a few key papers here:

%%%%%%% Vision application requires human input, not automatic enough. %%%%%%% 
Recently, advanced solutions based on computer vision
are proposed to mitigate the bottleneck
but are still limited by human processing bandwidth and far away from a fully automatic traffic surveillance.
% First, the learning-based computer vision algorithms require many hours of labeled data for model training.
First, while precise specifications of capturing camera or the starting and ending points play a large role in tracking accuracy \cite{yanziVehicleTracker,tamersoy2009robust,rodriguez2010adaptive,mishra2013video,cheng2011intelligent,corral2017slot}, 
the videos are time-consuming to hand-labeled accurately with this level of details,
while given a large number of cameras deployed nationwide, it is unwise to provide specifications for every camera.
%Although there are studies on entry/exit extraction \cite{tung2011goal, intawong2015detection, nedrich2013detecting}, and motion pattern learning \cite{wang2009unsupervised, kuettel2010s, hospedales2009markov}, with applications such as abnormality detection and semantic query, the application of 
%
Second,
any models trained for one scene cannot be deployed in other scenes,
as the semantics of surveillance scenes, such as entry/exit area \cite{tamersoy2009robust,rodriguez2010adaptive} or road surface \cite{bas2007automatic}, can differ from scene to scene. 
Also, the same camera can undergo slight adjustments in their focus and angle, and a model trained for that camera can become inaccurate and incur further costly re-calibrations.
%More generally, a model trained on labeled data from one camera can't be used to surveillance videos from another camera.
% , or to simply generalize models trained on a few cameras to all other cameras.
% DELETE COMMENT: Our method also needs training for each camera.

%%%%%%% Need hands-off real-time tracking without manual initialization and termination. %%%%%%% 
% Lastly but not the least, even after a model is trained and deployed,
% during real-time and non-stop surveillance while the videos streaming in,
% any dependency on human input for such knowledge will bottleneck a tracking model with the bandwidth of the human data processing.
% DELETE COMMENT: The prior knowledge is offline learned or set, we don't need to provide it again in real time processing once the system is set up.

%Last but not least, the manual annotation is not easily measured.
%Human's perception of the captured videos may greatly vary due to personal difference or concentration, which directly affect the annotation quality.
%the obtained semantic knowledge is still quite limited.

% start writing your own method.
In this paper, we aim at fully automatic semantic knowledge extraction from transportation videos and
significantly reduce human efforts in the vehicle tracking pipeline.
A vehicle tracker can benefit from the more restricted motion patterns in traffic videos and we 
propose a video scene semantic learning method to discover atomic motions, extent of active regions,
entry/exit hotspots, \etc..
We integrate the learned semantic knowledge into a tracker based on Kalman Filter,
which can flexibly accommodate state-of-the-art object detectors like faster-RCNN~\cite{renNIPS15fasterrcnn}.
Experiments on 13 long surveillance videos with diverse tracking scenes demonstrate the significantly improved tracking accuracy attained the proposed method, compared with other fully automatic approaches.

%%%%%%% Need apply scene learning to visual application %%%%%%% 

%%%%%%% Our solution %%%%%%% 
\setlist{nolistsep}
Our preliminary contributions are as follows:
\begin{itemize}%[noitemsep]
    % \setlength\itemsep{0em}
    \item Explicitly addressed those critical yet ignored issues for practical surveillance applications, such as automatic initialization and termination.
    \item Proposed a self-adaptive framework to where trackers are guided by the learned semantic scene knowledge, while in return the semantic knowledge is updated with tracking statistics.
    \item Provided a comprehensive evaluation of the proposed framework and demonstrate an significant improvement on object tracking.
\end{itemize}

%This paper is organized as follows: In Section \ref{sec:background}, we discuss the driving motivation of scene learning for surveillance applications. Section \ref{sec:sceneLearning} describes our method for extracting scene features from videos, which are subsequently used as prior knowledge to enhance tracking performance. In Section \ref{sec:evaluation}, we provide a comprehensive evaluation of both the learned scene knowledge and semantic knowledge aided tracker, which shows the learned semantic knowledge brings a big improvement to tracking frameworks. At last, we discuss some existing methods on surveillance vision application and scene learning in Section \ref{sec:related}. 

%\textcolor{red}{need to address scene update, shift the paper focus from tracking to scene understanding.}