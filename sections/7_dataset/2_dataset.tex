\section{Dataset}
To the best of our knowledge, there exists no public traffic surveillance video dataset containing complex real world interactions and illumination variations. 
Existing vision datasets are either not applicable to our scenario with different viewpoint (driver's view) \cite{sivaraman2010general} 
or contain short clips with limited adversarial conditions, scale changes, and illumination variations \cite{manen2014appearances,wu2015object}. %wang2009unsupervised
Even in the largest dataset collected \cite{wu2015object}, only 15 out of 98 videos exceeds 1000 frames (33 seconds).%, and only a single video containing 4,000 frames. 

We collected 13 representative traffic videos across our state, from the local department of transportation, and annotated these using VATIC \cite{springerlink:10.1007/s11263-012-0564-1}. Each object has its location and extent annotated on every frame, which is used as our ground truth. The average length of each video is five minutes (around 9000 frames), sufficient to cover several traffic signal cycles with real-world vehicle interactions and movement patterns. We divide the videos into two groups: simple low resolution (lowRes) and complex high resolution (highRes). \ref{fig:screenshots} shows screen shots from this dataset, and \ref{table:videos} gives an overview of our dataset, where the rightmost four columns indicate the number of videos reflecting various challenging aspects: occlusion, shadows, distortion and pedestrians.
%Although night videos also have high resolutions, we put then in a separate group due to the special illumination conditions.
%The dataset is currently available on [LINK].
\begin{table}[!htbp]
\footnotesize
\centering
\caption{Dataset overview. The second and the third columns show the resolution and object size range in pixels, followed by number of videos under each group. The rightmost four columns show the number of videos reflecting various challenging aspects (occlusion, shadow, distortion and pedestrian).}
\begin{tabular}{|c|c||c||c|c|c|c|c|}
    \hline
    Group & Resolution & Object size & \# & Occlusion & Shadow & Distortion & Pedestrian \\ \hline   
    \multirow{2}{*}{lowRes} & $342\times228$ & 32--44,814 & 5 & 3 & 1 & 3 & 0 \\ \cline{2-8}    
    ~                       & $320\times240$ & 48--25,284 & 2 & 2 & 1 & 2 & 0 \\ \hline
    highRes                 & $720\times576$ & 84--255,106 & 4 & 3 & 0 & 0 & 1 \\ \hline
%    night                   & $640\times 480$ & 81-134520 & 3 & 3 & 0 & 0 & 2 \\ \hline
\end{tabular}
% \vspace{-1em}
\label{table:videos}
\end{table}

\begin{figure}[!htbp]
\centering
    % \vspace{-1em}
    \begin{subfigure}{0.15\textwidth}
        \includegraphics[width=\linewidth, height = 0.7\linewidth]{./img/screenshots/{193402_Main_St_(US_51_Bus)_and_Empire_St_(IL_9)_in_Bloomington_20141023_11am}.png}
        \subcaption{}
        \label{subfig:193402}
    \end{subfigure}%
    \hspace{0.002\textwidth}
    \begin{subfigure}{0.15\textwidth}
        \includegraphics[width=\linewidth, height = 0.7\linewidth]{./img/screenshots/{intersection_4}.png}
        \subcaption{}
        \label{subfig:intersection_4}
    \end{subfigure}%
    \hspace{0.002\textwidth}
    \begin{subfigure}{0.15\textwidth}
        \includegraphics[width=\linewidth, height = 0.7\linewidth]{./img/screenshots/{251035_Princeton_34_&_26_T_20150812_08am}.png}
        \subcaption{}
        \label{subfig:251035}
    \end{subfigure}%
    \hspace{0.002\textwidth}
    \begin{subfigure}{0.15\textwidth}
        \includegraphics[width=\linewidth, height = 0.7\linewidth]{./img/screenshots/{243948_IL_126_@_Ridge_Rd._001_20150625_10am}.png}
        \subcaption{}
        \label{subfig:243948}
    \end{subfigure}%
    \hspace{0.002\textwidth}
    \begin{subfigure}{0.15\textwidth}
        \includegraphics[width=\linewidth, height = 0.7\linewidth]{./img/screenshots/{245837_FAI-74_E_of_St._Joseph_in_Champaign_County_20150630_09am}.png}
        \subcaption{}
        \label{subfig:245837}
    \end{subfigure}%
    \hspace{0.002\textwidth}
    \begin{subfigure}{0.15\textwidth}
        \includegraphics[width=\linewidth, height = 0.7\linewidth]{./img/screenshots/{252707_FAI-74_E_of_Lincoln_Ave_in_Urbana_20150826_09am}.png}
        \subcaption{}
        \label{subfig:252707}
    \end{subfigure}%
    \vspace{5pt}
    \begin{subfigure}{0.15\textwidth}
        \includegraphics[width=\linewidth, height = 0.7\linewidth]{./img/screenshots/{251950_IL_8_(E.Washington_St)_&_Illini_Dr_-_Farmdale_Rd_20150818_12pm}.png}
        \subcaption{}
        \label{subfig:251950}
    \end{subfigure}%
    \hspace{0.002\textwidth}
    % \begin{subfigure}{0.13\textwidth}
    %     \includegraphics[width=\linewidth, height = 0.7\linewidth]{./img/screenshots/{20150829_020000DST_ciceroPeterson}.png}
    %     \subcaption{}
    %     \label{subfig:ciceroPeterson}
    % \end{subfigure}
    % \hspace{0.002\textwidth}
    % \begin{subfigure}{0.13\textwidth}
    %     \includegraphics[width=\linewidth, height = 0.7\linewidth]{./img/screenshots/{20150829_020000DST_elstonIrvingPark}.png}
    %     \subcaption{}
    %     \label{subfig:elstonIrvingPark}
    % \end{subfigure}
    % \hspace{0.002\textwidth}
    \begin{subfigure}{0.15\textwidth}
        \includegraphics[width=\linewidth, height = 0.7\linewidth]{./img/screenshots/{ILCHI_CHI003_20151010_075033_051}.png}
        \subcaption{}
        \label{subfig:CHI003}
    \end{subfigure}%
    \hspace{0.002\textwidth}
    \begin{subfigure}{0.15\textwidth}
        \includegraphics[width=\linewidth, height = 0.7\linewidth]{./img/screenshots/{ILCHI_CHI120_20151013_095039_099}.png}
        \subcaption{}
        \label{subfig:CHI120}
    \end{subfigure}%
    \hspace{0.002\textwidth}
    \begin{subfigure}{0.15\textwidth}
        \includegraphics[width=\linewidth, height = 0.7\linewidth]{./img/screenshots/{ILCHI_CHI164_20150930_125029_234}.png}
        \subcaption{}
        \label{subfig:CHI164}
    \end{subfigure}%
    \hspace{0.002\textwidth}
    \begin{subfigure}{0.15\textwidth}
        \includegraphics[width=\linewidth, height = 0.7\linewidth]{./img/screenshots/{20150918_150500DST_halsted1}.png}
        \subcaption{}
        \label{subfig:halsted1}
    \end{subfigure}%
    \hspace{0.002\textwidth}
    \begin{subfigure}{0.15\textwidth}
        \includegraphics[width=\linewidth, height = 0.7\linewidth]{./img/screenshots/{20150918_150500DST_halsted2}.png}
        \subcaption{}
        \label{subfig:halsted2}
    \end{subfigure}
    % \vspace{-0.5em}
    \caption{Snapshots of videos in our dataset, with various resolution, viewpoint, illumination, vehicle size and interactions. In particular, (\subref{subfig:251035}) shows shadows; (\subref{subfig:252707}) and (\subref{subfig:251950}) show severe distortion by fish-eye camera. We group these videos by their characteristics: (\subref{subfig:193402}) - (\subref{subfig:251950}) are simple low resolution videos (lowRes), 
    %(\subref{subfig:ciceroPeterson}) and (\subref{subfig:elstonIrvingPark}) are night videos, 
    and (\subref{subfig:CHI003}) - (\subref{subfig:halsted2}) are complex high resolution videos (highRes).} %Note that the last two (\subref{subfig:halsted1}) and (\subref{subfig:halsted2}) are from the same video.}
    \label{fig:screenshots}
\end{figure}