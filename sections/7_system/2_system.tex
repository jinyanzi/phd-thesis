\section{Vehicle Counter}
\label{sec:system-counter}

After the tracking process finishes, the number of trajectories is regarded as vehicle counts.
However, the current traffic statistics of \gls{idot} contains specific vehicle counts on different directions. 
To match the format of the existing reports, not only we have to obtain the total number of vehicles, each vehicle has to be correctly classified to its corresponding motion.
Consequently, vehicle movements have to be available apart from the tracking results.
With different tracking strategy, motions are obtained differently: with our heuristic tracker, we rely on human annotated input; however, with the semantic tracker, we learn the motion offline by unsupervised learning.

\subsection{Vehicle Counter with Human Annotation}
Initially, users upload videos via FTP and operated on our GUI interface by remote desktop.
Our GUI interface and require the user to draw a few line segments as the templates of the vehicle motion.
\ref{fig:anno-gui} is the screen shot of the interface, where the motion templates mostly align with the road surface. 
However, a road may have more than one motion due to potential multiple lanes. 
\ref{fig:kf-counter} shows our first counter framework with the heuristic tracker. 
The tracker generates a set of vehicle trajectories, then each trajectory is assigned to a motion template that mostly matches its movement. 
By increasing the count of each vehicle's assigned motion template, we can obtain the final traffic count of each movement. 

Suppose we have a set of $n$ motion templates $\mathbf{T} = \{\mathbf{T}_1, \mathbf{T}_2, \dots, \mathbf{T}_n\}$ and a set of trajectories of $N$ vehicles $\Omega= \{\mathbf{O}_1, \mathbf{O}_2, \dots, \mathbf{O}_N\}$.

\begin{figure}
\centering
\includegraphics[width=\linewidth]{./img/system/kf-counter.pdf}
\caption{Vehicle counter workflow with human annotation.}
\label{fig:kf-counter}
\end{figure}
\begin{figure}
\centering
\includegraphics[width=\linewidth]{./img/system/line_seg.png}
\caption{Motion annotation interface.}
\label{fig:anno-gui}
\end{figure}

\subsection{Vehicle Counter with Semantic Knowledge}
With offline learned semantic knowledge per camera, the system becomes end-to-end without any human input. 
\ref{fig:semantic-counter} shows the workflow of the improved counter with semantic knowledge. 
Motion topics are learned fully automatically.
By online inference, each tracked vehicle is assigned to a motion until leaving, therefore, the counting is finished along tracking. 

\begin{figure}
\centering
\includegraphics[width=\linewidth]{./img/system/semantic-counter.pdf}
\caption{End-to-end vehicle counter workflow with scene understanding.}
\label{fig:semantic-counter}
\end{figure}

\subsection{Web Portal}
For remote access, we build a web portal for this system for \gls{idot}. Currently the scene understanding module has not been integrated, only the what shown in \ref{fig:kf-counter} is implemented.
Cameras are displayed on map as shown in \ref{fig:sys-main}, with a list of their name on the left. Users may create or browse cameras by list or map, and upload videos for each camera. Once a new camera is created and the first video is uploaded, the user need to draw the motion templates.
Once a video is uploaded, the tracker and counter will be called in the background. Users may return later to check and download the results. We allow at most four videos processed at the same time. 
\ref{fig:sys-camera} and \ref{fig:sys-video} are the camera view and video view with traffic summary and visualization. 
Each camera may have multiple videos. Summary of all the videos is displayed. In the video view, counts of different motions are displayed.
\begin{figure}
    \centering
    \includegraphics[width=\linewidth]{./img/system/main.png}
    \caption{Main interface of the web portal, cameras are displayed on the map.}
    \label{fig:sys-main}
\end{figure}
\begin{figure}
    \centering
    \includegraphics[width=\linewidth]{./img/system/camera.png}
    \caption{Camera view with video list and summary.}
    \label{fig:sys-camera}
\end{figure}
\begin{figure}
    \centering
    \includegraphics[width=\linewidth]{./img/system/video.png}
    \caption{Individual video view with counting information.}
    \label{fig:sys-video}
\end{figure}
