\section{Introduction}
\label{sec:sys-intro}

People in civil engineering are devoted to building a better social environment.
As one of its sub-discipline, traffic engineering focuses on efficient traffic flow. 
It aims to achieve safe and efficient movement of people and goods on existing transportation infrastructures. 
For example, by properly designing road geometry, the average travel time may be shortened; 
by modification of the traffic lights and signs, the crash rate at a specific location can decrease significantly.

To make the right decision, sufficient data should be acquired to support quantitative analysis.
Under the theory of traffic engineering, the famous Lane flow equation \cite{roess2004traffic} describes the relationship between traffic flow and speed:
$$Q = KV,$$
where $Q$ is the number of vehicles per hour, $V$ is the mean speed, $K$ is the vehicle density, usually can be changed by the speed limit, signals on the ramp entrance. 

In general, we want our facilities to have maximal flow capacity, and $Q$ is the quantity of interest.
Therefore, the traffic flow $Q$ needs to be obtained to evaluate the changes reflected by the decision made to the infrastructures.
Apart from the traditional heavy equipment, increasing attention and efforts have been put on analyzing the existing traffic videos.
For example, \gls{idot} maintains a huge database of videos recorded by traffic cameras across Illinois 24/7. 
People are hired to manually count the number of vehicles and generate a report for each one-hour video, which is expensive both in labor and time.
To help facilitate the process, we build an end-to-end vehicle counting system on top of our fully automatic tracker. 
The system runs in real time, generates a similar report for each video. It significantly reduces the cost and time for this process.

In practice, the tracking and counting process requires immense computational resource; therefore, these tasks usually run on remote servers. 
To allow easy access to the data and results, we build additional GUI tools to provide the users interactive operation, such as video upload, result visualization, and report generation.
With end-to-end workflow and interactive GUI interface, the system can be easily deployed in large scale.