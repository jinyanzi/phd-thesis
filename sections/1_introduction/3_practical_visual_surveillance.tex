\section{Practical visual surveillance}
\label{sec:intro-surveillance-cv}

While people in both civil engineering and computer vision are working toward the same goal --- building the real-world visual surveillance systems, there is still a big gap between the state-of-the-art research and practice use.
Researchers in academia need an in-depth study of a specific problem; therefore, they usually make simplified assumptions to isolate the problem and ignore the impact of other factors. 
However, the real-world application runs in a complex environment and deals with noisy data. Data cleaning and processing speed have to be taken into account.
For example, researchers care more about the novelty and performance of the solution and assume the computation resources are unlimited. 
In this case, even though some algorithms outperform human, they are too slow to process a massive amount of data. 
Besides, academic problems usually have a standard benchmark with well-processed data, so that people can spend the minimal amount of time on evaluation, with a uniformly acknowledged standard.
However, those data might be too small and ideal compared with the real-world data, and the evaluation metric may not be comprehensive for those complex and noisy data for different tasks.

After some failed trials of computer vision algorithms on data from the local department of transportation, we found out that noisy data is one major obstacle for processing real-world data. 
Besides, minimal human input and real-time processing speed is the prerequisite for large scale application.
On the other hand, there lacks a standard benchmark and dataset for the researchers in the field to study the visual surveillance problems and solve the bottleneck problem.

In this thesis, we try to address the aforementioned problems neglected in visual surveillance field and narrow the gap between the computer vision research and large scale application. 
Specifically, we are building an end-to-end pipeline for transportation video analysis, minimizing human input and maximizing the throughput.
We aim to solve the problems in vehicle tracking and counting that are critical to the performance, such as automatic tracking initialization/termination, noise elimination.
On top of that, we try to further improve the algorithm performance by learning semantic knowledge in an unsupervised manner, taking into account that the regular vehicle motion pattern from a static camera  can be informative for analytic tasks.
Through comprehensive experiment and case studies, we show that proper initialization and termination improves the performance of the general automatic tracking framework with any tracking algorithm; and the semantic knowledge brings more benefits to multiple visual surveillance tasks.
As a by-product of our experiment, we annotate and release a large dataset for the community \cite{yanziVehicleTracker}, aiming to attract and help more researchers to work on such problems.