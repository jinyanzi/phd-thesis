\section{Practical challenges}
\label{sec:intro-cv}

While people in both civil engineering and computer vision are working toward the same goal --- building the real-world intelligent transportation system via computer vision, there is still a big gap between the state-of-the-art research and practice use.
Researchers in academia need an in-depth study of a specific problem; therefore, they usually make simplified assumptions to isolate the problem and ignore the impact of other factors. 
However, the real-world application runs in a complex environment and deals with noisy data. Data cleaning and processing speed have to be taken into account.
For example, researchers care more about the novelty and performance of the solution and assume the computation resources are unlimited. 
In this case, even though some algorithms outperform human, they are too slow to process a massive amount of data. 
Besides, academic problems usually have a standard benchmark with well-processed data, so that people can spend the minimal amount of time on evaluation, with a uniformly acknowledged standard.
However, those data might be too small and ideal compared with the real-world data, and the evaluation metric may not be comprehensive for those complex and noisy data for different tasks.

% After some failed trials of computer vision algorithms on data from the local department of transportation, 
We found out through experiment that noisy data is one major obstacle for processing real-world videos. 
Besides, minimal human input and real-time processing speed is the prerequisite for large scale application.
On the other hand, there lacks a standard benchmark and dataset for the researchers in the field to study the visual transportation problems and solve the bottleneck problem.
