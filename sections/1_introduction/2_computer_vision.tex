\section{Computer vision on surveillance data}
\label{sec:intro-cv}

With the progress in artificial intelligence and higher demand in its application, computer vision is becoming one of its hottest sub-domains. 
Computer vision enables computers to perform tasks that human is good at, such as visual detection and tracking. 
With the superior advantage of computational speed, these tasks can be applied in large scale.

When the fundamental theory of this field has become rather mature, researchers gradually shift their attention to more specific tasks and data. For example, pedestrian detection \cite{dollar2012pedestrian} and face recognition \cite{parkhi2015deep}, gesture recognition \cite{rautaray2015vision} has become individual topics due to their huge application potential and high demand.
On the other hand, surveillance videos have also drawn much attention because of the various practical challenges and the huge impact of applications. 
Researchers are studying the computer vision problems specific to surveillance videos, such as anomaly detection \cite{scime2018anomaly}, tracking \cite{wu2015object} and counting \cite{seenouvong2016computer}. 


The surveillance videos have a few unique features: 
\begin{itemize}
\item they are recorded ceaselessly, therefore, in large quantities; 
\item they are usually of low-resolution qualities due to the transmission and storage limits;
\item they have a special composition (pedestrian and vehicle) and the objects in the view move in a regular motion pattern.
\item depending on the location, the video may contain a large number of objects with highly complex interactions.
\end{itemize}

Therefore, the complexity of the videos and the high-performance requirement raise challenges for robust surveillance systems. 
Ideally, they are expected to perform the task with one-time setup and the minimal amount of human effort, with high accuracy and high speed.