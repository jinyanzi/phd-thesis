\section{Surveillance monitoring via computer vision}
\label{sec:intro-surveillance}

Surveillance tasks are an essential source of decision making for monitoring and management purpose.
A private commercial shopping mall may need the statistics of customer visit count and average shopping time to analyze sale performance; 
surveillance record may provide evidence to the police and help to solve criminal cases;
people civil engineering department may need the data of traffic flow and speed for better traffic signal design.
Hand-held counting device or written record at the interested location were widely used to keep track of people.
For vehicles on the road, sizeable physical equipment was mainly used, such as pressure tubes laid across the pavement, magnetic loops under the pavement \cite{klein2006traffic,mimbela2000summary}. 
These equipment are usually hard to set up and maintain; therefore are generally labor expensive.

With the progress of the hardware, the cost of cameras has been significantly reduced. 
Compared with the traditional heavy and inefficient equipment, they are more lightweight and easier to set up. 
Due to the low cost and full coverage of the surveillance area, surveillance cameras are prevalently installed all across the city and run continuously.
The following abundant video data demand efficient indexing and information extraction for surveillance tasks, involving interdisciplinary research such database and video/image processing.

