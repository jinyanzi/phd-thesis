\section{Traffic engineering}
\label{sec:intro-its}

% Surveillance infrastructures generate vast amounts of data which, if effectively analyzed, provide huge potential for informing decision making.
% A shopping mall may use the statistics of customer visit count and average shopping time to analyze sales performance; 
% surveillance records may provide evidence to the police and help solve criminal cases;
% urban planners may use traffic flow data to develop better traffic patterns.
Vehicles on the road generate vast amounts of data which, if effectively analyzed, provide huge potential for informing decision making.
In traffic engineering, urban planners use traffic flow data to develop better traffic patterns.
The standard method for collecting data involves manual counting people by a field agent.
For vehicles on the road, equipment such as pressure tubes laid across the pavement, magnetic loops under the pavement \cite{klein2006traffic,mimbela2000summary} is used to collect data, but it is hard and expensive to set up and maintain.

On the other hand, the cost of cameras continues to decrease, therefore, deploying a camera system rather than traditional equipment becomes an obvious choice.
The wide deployment of cameras leads to better coverage of the city, but also a massive amount of video data.
The essential task for image-based data collection is extracting and efficient indexing of desired data.
Currently, such tasks still primarily rely on eye-balling the interested object or event, which is inefficient.
Fortunately, computer vision is becoming powerful and approaching human performance on some problems.
People tend to seek more cost-efficient solution of traffic flow analysis via computer vision.
With the computational advantage, computer vision techniques are expected to cut the cost and scale up those tasks.